%%%%%%%%%%%%%%%%%%%%%%%%%%%%%%%%%%%%%%%%%%%%%%%%%%%%%%%%%%%%%%%%%%%%%%%%%%%%%%%%%%%%
% Do not alter this block (unless you're familiar with LaTeX
\documentclass{article}
\usepackage[margin=1in]{geometry} 
\usepackage{amsmath,amsthm,amssymb,amsfonts, fancyhdr, color, comment, graphicx, environ}
\usepackage{xcolor}
\usepackage{mdframed}
\usepackage{CJKutf8}
\usepackage[colorlinks, linkcolor=blue]{hyperref}
\usepackage[shortlabels]{enumitem}
\usepackage{indentfirst}
\usepackage{hyperref}
\hypersetup{
    colorlinks=true,
    linkcolor=blue,
    filecolor=magenta,      
    urlcolor=blue,
}


\pagestyle{fancy}


\newenvironment{problem}[2][Problem]
    { \begin{mdframed}[backgroundcolor=gray!20] \textbf{#1 #2} \\}
    {  \end{mdframed}}

\newenvironment{question}[2][Question]
    { \begin{mdframed}[backgroundcolor=gray!5] \textbf{#1 #2} \\}
    {  \end{mdframed}}

% Define solution environment
\newenvironment{solution}{\textbf{Solution}}

%%%%%%%%%%%%%%%%%%%%%%%%%%%%%%%%%%%%%%%%%%%%%
%Fill in the appropriate information below
\lhead{Date: 2021/10/30}
\chead{\textbf{HW2}}
\rhead{1179 Probability} 
%%%%%%%%%%%%%%%%%%%%%%%%%%%%%%%%%%%%%%%%%%%%%


\begin{document}
\begin{CJK*}{UTF8}{bsmi}
Name: 陳品劭 \qquad ID: 109550206 \qquad
\href{https://www.overleaf.com/read/kyjmbmnjwxzb}{Self link} \qquad
\href{https://www.overleaf.com/read/fmqrvpdhwqpg}{Others link}
    \begin{problem}{5 (a)}
    
    \end{problem}
    
    \begin{solution}
    
    $\lambda > 0$
    
    We know the CDF of $X$ is $F_X(t)=\{1-e^{-\lambda t}$, if $t\ge0$; $0$, otherwise.
    
    Then $Y=aX+b$.
    
    The CDF of $Y$ is $F_Y(t)=P(Y\le t)=P(aX+b\le t)=P(X\le \frac{t-b}{a})=F_X(\frac{t-b}{a})=\{1-e^{-\lambda \frac{t-b}{a}}$, if $\frac{t-b}{a}\ge0$; $0$ otherwise.
    \newline
    
    $\frac{d}{dt}(1-e^{-\lambda \frac{t-b}{a}})=\frac{\lambda}{a}e^{-\lambda \frac{t-b}{a}}$.
    
    The PDF of $Y$ is  $f_Y(t)=\{\frac{\lambda}{a}e^{-\lambda \frac{t-b}{a}}$, if $\frac{t-b}{a}\ge0$; $0$ otherwise.
    \newline
    
    Define $\lambda^*=\frac{\lambda}{a}$.
    
    $\lambda^*$ should greater than 0 and $t$ have interval $[0,\infty)$.
    
    Hence $a>0$ and $b=0$.
    
    When $a>0$ and $b=0$, $Y$ also is an exponential random variable:
    
    
    \end{solution}
    
    \begin{problem}{5 (b)}
    
    \end{problem}
    
    \begin{solution}
    
    The PDF of standard normal variables is $P(X=x)=\frac{1}{2\pi}\exp(\frac{-x^2}{2})$, $\forall x \in \mathbb{R}$.
    \newline
    
    $E[X]=\int_{-\infty}^{\infty}{x\frac{1}{\sqrt{2\pi}}e^{\frac{-x^2}{2}}dx}=\frac{1}{\sqrt{2\pi}}[e^{-x^2/2}]^{\infty}_{-\infty}=\frac{-1}{\sqrt{2\pi}}(\lim\limits_{t\rightarrow{\infty}}[e^{-x^2/2}]_{-t}^{0}+\lim\limits_{t\rightarrow{\infty}}[e^{-x^2/2}]_{0}^{t})=\frac{-1}{\sqrt{2\pi}} \times 0=0$
    \newline
    
    $Var[X]=\int_{-\infty}^{\infty}{(x-E[X])^2\frac{1}{\sqrt{2\pi}}e^{-x^2/2}dx}=Var[X]=\frac{1}{\sqrt{2\pi}}\int_{-\infty}^{\infty}{(x)^2e^{-x^2/2}dx}$
    
    $\hspace*{36}
    =\frac{1}{\sqrt{2\pi}}([-xe^{-x^2/2}]^{\infty}_{-\infty}+\int_{-\infty}^{\infty}e^{-x^2/2}dx)=\frac{1}{\sqrt{2\pi}}(0+\int_{-\infty}^{\infty}e^{-x^2/2}dx)=\frac{1}{\sqrt{2\pi}}\times \sqrt{2\pi}=1$
    \newline
    
    Verified that a standard normal random variable $X$ satisfies that $Var[X] = 1$.
    \end{solution}
    
\end{CJK*}
\end{document}
